% Options for packages loaded elsewhere
\PassOptionsToPackage{unicode}{hyperref}
\PassOptionsToPackage{hyphens}{url}
%
\documentclass[
]{article}
\usepackage{amsmath,amssymb}
\usepackage{iftex}
\ifPDFTeX
  \usepackage[T1]{fontenc}
  \usepackage[utf8]{inputenc}
  \usepackage{textcomp} % provide euro and other symbols
\else % if luatex or xetex
  \usepackage{unicode-math} % this also loads fontspec
  \defaultfontfeatures{Scale=MatchLowercase}
  \defaultfontfeatures[\rmfamily]{Ligatures=TeX,Scale=1}
\fi
\usepackage{lmodern}
\ifPDFTeX\else
  % xetex/luatex font selection
\fi
% Use upquote if available, for straight quotes in verbatim environments
\IfFileExists{upquote.sty}{\usepackage{upquote}}{}
\IfFileExists{microtype.sty}{% use microtype if available
  \usepackage[]{microtype}
  \UseMicrotypeSet[protrusion]{basicmath} % disable protrusion for tt fonts
}{}
\makeatletter
\@ifundefined{KOMAClassName}{% if non-KOMA class
  \IfFileExists{parskip.sty}{%
    \usepackage{parskip}
  }{% else
    \setlength{\parindent}{0pt}
    \setlength{\parskip}{6pt plus 2pt minus 1pt}}
}{% if KOMA class
  \KOMAoptions{parskip=half}}
\makeatother
\usepackage{xcolor}
\usepackage[margin=1in]{geometry}
\usepackage{color}
\usepackage{fancyvrb}
\newcommand{\VerbBar}{|}
\newcommand{\VERB}{\Verb[commandchars=\\\{\}]}
\DefineVerbatimEnvironment{Highlighting}{Verbatim}{commandchars=\\\{\}}
% Add ',fontsize=\small' for more characters per line
\usepackage{framed}
\definecolor{shadecolor}{RGB}{248,248,248}
\newenvironment{Shaded}{\begin{snugshade}}{\end{snugshade}}
\newcommand{\AlertTok}[1]{\textcolor[rgb]{0.94,0.16,0.16}{#1}}
\newcommand{\AnnotationTok}[1]{\textcolor[rgb]{0.56,0.35,0.01}{\textbf{\textit{#1}}}}
\newcommand{\AttributeTok}[1]{\textcolor[rgb]{0.13,0.29,0.53}{#1}}
\newcommand{\BaseNTok}[1]{\textcolor[rgb]{0.00,0.00,0.81}{#1}}
\newcommand{\BuiltInTok}[1]{#1}
\newcommand{\CharTok}[1]{\textcolor[rgb]{0.31,0.60,0.02}{#1}}
\newcommand{\CommentTok}[1]{\textcolor[rgb]{0.56,0.35,0.01}{\textit{#1}}}
\newcommand{\CommentVarTok}[1]{\textcolor[rgb]{0.56,0.35,0.01}{\textbf{\textit{#1}}}}
\newcommand{\ConstantTok}[1]{\textcolor[rgb]{0.56,0.35,0.01}{#1}}
\newcommand{\ControlFlowTok}[1]{\textcolor[rgb]{0.13,0.29,0.53}{\textbf{#1}}}
\newcommand{\DataTypeTok}[1]{\textcolor[rgb]{0.13,0.29,0.53}{#1}}
\newcommand{\DecValTok}[1]{\textcolor[rgb]{0.00,0.00,0.81}{#1}}
\newcommand{\DocumentationTok}[1]{\textcolor[rgb]{0.56,0.35,0.01}{\textbf{\textit{#1}}}}
\newcommand{\ErrorTok}[1]{\textcolor[rgb]{0.64,0.00,0.00}{\textbf{#1}}}
\newcommand{\ExtensionTok}[1]{#1}
\newcommand{\FloatTok}[1]{\textcolor[rgb]{0.00,0.00,0.81}{#1}}
\newcommand{\FunctionTok}[1]{\textcolor[rgb]{0.13,0.29,0.53}{\textbf{#1}}}
\newcommand{\ImportTok}[1]{#1}
\newcommand{\InformationTok}[1]{\textcolor[rgb]{0.56,0.35,0.01}{\textbf{\textit{#1}}}}
\newcommand{\KeywordTok}[1]{\textcolor[rgb]{0.13,0.29,0.53}{\textbf{#1}}}
\newcommand{\NormalTok}[1]{#1}
\newcommand{\OperatorTok}[1]{\textcolor[rgb]{0.81,0.36,0.00}{\textbf{#1}}}
\newcommand{\OtherTok}[1]{\textcolor[rgb]{0.56,0.35,0.01}{#1}}
\newcommand{\PreprocessorTok}[1]{\textcolor[rgb]{0.56,0.35,0.01}{\textit{#1}}}
\newcommand{\RegionMarkerTok}[1]{#1}
\newcommand{\SpecialCharTok}[1]{\textcolor[rgb]{0.81,0.36,0.00}{\textbf{#1}}}
\newcommand{\SpecialStringTok}[1]{\textcolor[rgb]{0.31,0.60,0.02}{#1}}
\newcommand{\StringTok}[1]{\textcolor[rgb]{0.31,0.60,0.02}{#1}}
\newcommand{\VariableTok}[1]{\textcolor[rgb]{0.00,0.00,0.00}{#1}}
\newcommand{\VerbatimStringTok}[1]{\textcolor[rgb]{0.31,0.60,0.02}{#1}}
\newcommand{\WarningTok}[1]{\textcolor[rgb]{0.56,0.35,0.01}{\textbf{\textit{#1}}}}
\usepackage{graphicx}
\makeatletter
\def\maxwidth{\ifdim\Gin@nat@width>\linewidth\linewidth\else\Gin@nat@width\fi}
\def\maxheight{\ifdim\Gin@nat@height>\textheight\textheight\else\Gin@nat@height\fi}
\makeatother
% Scale images if necessary, so that they will not overflow the page
% margins by default, and it is still possible to overwrite the defaults
% using explicit options in \includegraphics[width, height, ...]{}
\setkeys{Gin}{width=\maxwidth,height=\maxheight,keepaspectratio}
% Set default figure placement to htbp
\makeatletter
\def\fps@figure{htbp}
\makeatother
\setlength{\emergencystretch}{3em} % prevent overfull lines
\providecommand{\tightlist}{%
  \setlength{\itemsep}{0pt}\setlength{\parskip}{0pt}}
\setcounter{secnumdepth}{-\maxdimen} % remove section numbering
\ifLuaTeX
  \usepackage{selnolig}  % disable illegal ligatures
\fi
\usepackage{bookmark}
\IfFileExists{xurl.sty}{\usepackage{xurl}}{} % add URL line breaks if available
\urlstyle{same}
\hypersetup{
  pdftitle={GV300 Showcase: A/B Testing in Political Communication},
  hidelinks,
  pdfcreator={LaTeX via pandoc}}

\title{GV300 Showcase: A/B Testing in Political Communication}
\author{}
\date{\vspace{-2.5em}}

\begin{document}
\maketitle

\subsection{Introduction}\label{introduction}

For my GV300 assignment on A/B testing, I simulated an experiment
testing whether an issue-focused political homepage increases supporter
signups compared to standard party messaging. This analysis demonstrates
core causal inference concepts including random assignment, potential
outcomes, and treatment effect estimation.

The experiment randomly assigned 1,500 visitors to either a control
homepage (standard messaging) or treatment homepage (emphasizing a
hot-button issue like immigration). The outcome of interest was whether
visitors signed up as party supporters.

\begin{center}\rule{0.5\linewidth}{0.5pt}\end{center}

\subsection{Code Example: Simulating the
Experiment}\label{code-example-simulating-the-experiment}

Below is the R code I used to generate the experimental data. This
simulation creates potential outcomes under both conditions and then
reveals only the outcome corresponding to each visitor's actual
treatment assignment.

\begin{Shaded}
\begin{Highlighting}[]
\CommentTok{\# Load required libraries}
\FunctionTok{library}\NormalTok{(ggplot2)}
\FunctionTok{library}\NormalTok{(dplyr)}

\CommentTok{\# Set seed for reproducibility}
\FunctionTok{set.seed}\NormalTok{(}\DecValTok{7263}\NormalTok{)}

\CommentTok{\# 1. Sample size}
\NormalTok{n }\OtherTok{\textless{}{-}} \DecValTok{1500}

\CommentTok{\# 2. Random treatment assignment (50/50 split)}
\NormalTok{treatment }\OtherTok{\textless{}{-}} \FunctionTok{rbinom}\NormalTok{(}\AttributeTok{n =}\NormalTok{ n, }\AttributeTok{size =} \DecValTok{1}\NormalTok{, }\AttributeTok{prob =} \FloatTok{0.5}\NormalTok{)}

\CommentTok{\# 3. Define potential outcomes}
\NormalTok{p\_control }\OtherTok{\textless{}{-}} \FloatTok{0.12}      \CommentTok{\# Baseline signup probability}
\NormalTok{p\_treated }\OtherTok{\textless{}{-}} \FloatTok{0.17}      \CommentTok{\# Treatment increases by 5 percentage points}

\NormalTok{Y0 }\OtherTok{\textless{}{-}} \FunctionTok{rbinom}\NormalTok{(}\AttributeTok{n =}\NormalTok{ n, }\AttributeTok{size =} \DecValTok{1}\NormalTok{, }\AttributeTok{prob =}\NormalTok{ p\_control)  }\CommentTok{\# Outcome under control}
\NormalTok{Y1 }\OtherTok{\textless{}{-}} \FunctionTok{rbinom}\NormalTok{(}\AttributeTok{n =}\NormalTok{ n, }\AttributeTok{size =} \DecValTok{1}\NormalTok{, }\AttributeTok{prob =}\NormalTok{ p\_treated)  }\CommentTok{\# Outcome under treatment}

\CommentTok{\# 4. Realised outcome (depends on actual treatment received)}
\NormalTok{support\_signup }\OtherTok{\textless{}{-}} \FunctionTok{ifelse}\NormalTok{(treatment }\SpecialCharTok{==} \DecValTok{1}\NormalTok{, Y1, Y0)}

\CommentTok{\# 5. Create data frame}
\NormalTok{party\_ab }\OtherTok{\textless{}{-}} \FunctionTok{data.frame}\NormalTok{(}
  \AttributeTok{visitor\_id =} \DecValTok{1}\SpecialCharTok{:}\NormalTok{n,}
  \AttributeTok{treatment =}\NormalTok{ treatment,}
  \AttributeTok{support\_signup =}\NormalTok{ support\_signup}
\NormalTok{)}

\CommentTok{\# Display first few rows}
\FunctionTok{head}\NormalTok{(party\_ab)}
\end{Highlighting}
\end{Shaded}

\begin{verbatim}
##   visitor_id treatment support_signup
## 1          1         0              1
## 2          2         1              0
## 3          3         1              0
## 4          4         1              0
## 5          5         1              0
## 6          6         0              0
\end{verbatim}

The key conceptual point illustrated here is the \textbf{fundamental
problem of causal inference}: for each visitor, we only observe one
potential outcome (Y0 or Y1) based on which homepage they actually saw.
The counterfactual---what would have happened under the other
condition---remains unobserved.

\begin{center}\rule{0.5\linewidth}{0.5pt}\end{center}

\subsection{Data Inspection}\label{data-inspection}

\begin{Shaded}
\begin{Highlighting}[]
\CommentTok{\# Summary statistics}
\FunctionTok{summary}\NormalTok{(party\_ab)}
\end{Highlighting}
\end{Shaded}

\begin{verbatim}
##    visitor_id       treatment     support_signup  
##  Min.   :   1.0   Min.   :0.000   Min.   :0.0000  
##  1st Qu.: 375.8   1st Qu.:0.000   1st Qu.:0.0000  
##  Median : 750.5   Median :1.000   Median :0.0000  
##  Mean   : 750.5   Mean   :0.528   Mean   :0.1367  
##  3rd Qu.:1125.2   3rd Qu.:1.000   3rd Qu.:0.0000  
##  Max.   :1500.0   Max.   :1.000   Max.   :1.0000
\end{verbatim}

\begin{Shaded}
\begin{Highlighting}[]
\CommentTok{\# Check treatment balance}
\FunctionTok{table}\NormalTok{(party\_ab}\SpecialCharTok{$}\NormalTok{treatment)}
\end{Highlighting}
\end{Shaded}

\begin{verbatim}
## 
##   0   1 
## 708 792
\end{verbatim}

\begin{Shaded}
\begin{Highlighting}[]
\FunctionTok{prop.table}\NormalTok{(}\FunctionTok{table}\NormalTok{(party\_ab}\SpecialCharTok{$}\NormalTok{treatment))}
\end{Highlighting}
\end{Shaded}

\begin{verbatim}
## 
##     0     1 
## 0.472 0.528
\end{verbatim}

\begin{Shaded}
\begin{Highlighting}[]
\CommentTok{\# Signup rates by group}
\FunctionTok{aggregate}\NormalTok{(support\_signup }\SpecialCharTok{\textasciitilde{}}\NormalTok{ treatment, }\AttributeTok{data =}\NormalTok{ party\_ab, mean)}
\end{Highlighting}
\end{Shaded}

\begin{verbatim}
##   treatment support_signup
## 1         0      0.0960452
## 2         1      0.1729798
\end{verbatim}

The random assignment worked well: approximately 50\% of visitors were
assigned to each condition. The control group shows a signup rate around
\textbf{9.6\%}, while the treatment group shows approximately
\textbf{17.3\%}---a difference of \textbf{7.7 percentage points}.

\begin{center}\rule{0.5\linewidth}{0.5pt}\end{center}

\subsection{Visualisation: Treatment
Effects}\label{visualisation-treatment-effects}

My favourite visualisation from this assignment shows the signup rates
for both groups with confidence intervals, making the uncertainty around
the estimates visible to readers.

\begin{Shaded}
\begin{Highlighting}[]
\CommentTok{\# Calculate group means and standard errors}
\NormalTok{group\_summary }\OtherTok{\textless{}{-}}\NormalTok{ party\_ab }\SpecialCharTok{\%\textgreater{}\%}
  \FunctionTok{group\_by}\NormalTok{(treatment) }\SpecialCharTok{\%\textgreater{}\%}
  \FunctionTok{summarise}\NormalTok{(}
    \AttributeTok{mean\_signup =} \FunctionTok{mean}\NormalTok{(support\_signup),}
    \AttributeTok{se =} \FunctionTok{sqrt}\NormalTok{(mean\_signup }\SpecialCharTok{*}\NormalTok{ (}\DecValTok{1} \SpecialCharTok{{-}}\NormalTok{ mean\_signup) }\SpecialCharTok{/} \FunctionTok{n}\NormalTok{()),}
    \AttributeTok{.groups =} \StringTok{"drop"}
\NormalTok{  ) }\SpecialCharTok{\%\textgreater{}\%}
  \FunctionTok{mutate}\NormalTok{(}
    \AttributeTok{treatment\_label =} \FunctionTok{ifelse}\NormalTok{(treatment }\SpecialCharTok{==} \DecValTok{0}\NormalTok{, }\StringTok{"Control"}\NormalTok{, }\StringTok{"Issue{-}focused"}\NormalTok{),}
    \AttributeTok{ci\_lower =}\NormalTok{ mean\_signup }\SpecialCharTok{{-}} \FloatTok{1.96} \SpecialCharTok{*}\NormalTok{ se,}
    \AttributeTok{ci\_upper =}\NormalTok{ mean\_signup }\SpecialCharTok{+} \FloatTok{1.96} \SpecialCharTok{*}\NormalTok{ se}
\NormalTok{  )}

\CommentTok{\# Create the plot}
\FunctionTok{ggplot}\NormalTok{(group\_summary, }\FunctionTok{aes}\NormalTok{(}\AttributeTok{x =}\NormalTok{ treatment\_label, }\AttributeTok{y =}\NormalTok{ mean\_signup, }\AttributeTok{fill =}\NormalTok{ treatment\_label)) }\SpecialCharTok{+}
  \FunctionTok{geom\_col}\NormalTok{(}\AttributeTok{width =} \FloatTok{0.6}\NormalTok{, }\AttributeTok{alpha =} \FloatTok{0.85}\NormalTok{) }\SpecialCharTok{+}
  \FunctionTok{geom\_errorbar}\NormalTok{(}
    \FunctionTok{aes}\NormalTok{(}\AttributeTok{ymin =}\NormalTok{ ci\_lower, }\AttributeTok{ymax =}\NormalTok{ ci\_upper),}
    \AttributeTok{width =} \FloatTok{0.15}\NormalTok{,}
    \AttributeTok{linewidth =} \FloatTok{0.8}
\NormalTok{  ) }\SpecialCharTok{+}
  \FunctionTok{geom\_text}\NormalTok{(}
    \FunctionTok{aes}\NormalTok{(}\AttributeTok{label =} \FunctionTok{paste0}\NormalTok{(}\FunctionTok{round}\NormalTok{(mean\_signup }\SpecialCharTok{*} \DecValTok{100}\NormalTok{, }\DecValTok{1}\NormalTok{), }\StringTok{"\%"}\NormalTok{)),}
    \AttributeTok{vjust =} \SpecialCharTok{{-}}\FloatTok{0.7}\NormalTok{,}
    \AttributeTok{size =} \DecValTok{5}
\NormalTok{  ) }\SpecialCharTok{+}
  \FunctionTok{scale\_y\_continuous}\NormalTok{(}
    \AttributeTok{limits =} \FunctionTok{c}\NormalTok{(}\DecValTok{0}\NormalTok{, }\FloatTok{0.25}\NormalTok{),}
    \AttributeTok{labels =}\NormalTok{ scales}\SpecialCharTok{::}\FunctionTok{percent\_format}\NormalTok{(}\AttributeTok{accuracy =} \DecValTok{1}\NormalTok{),}
    \AttributeTok{expand =} \FunctionTok{expansion}\NormalTok{(}\AttributeTok{mult =} \FunctionTok{c}\NormalTok{(}\DecValTok{0}\NormalTok{, }\FloatTok{0.05}\NormalTok{))}
\NormalTok{  ) }\SpecialCharTok{+}
  \FunctionTok{scale\_fill\_manual}\NormalTok{(}\AttributeTok{values =} \FunctionTok{c}\NormalTok{(}\StringTok{"Control"} \OtherTok{=} \StringTok{"gray70"}\NormalTok{, }\StringTok{"Issue{-}focused"} \OtherTok{=} \StringTok{"steelblue"}\NormalTok{)) }\SpecialCharTok{+}
  \FunctionTok{labs}\NormalTok{(}
    \AttributeTok{x =} \ConstantTok{NULL}\NormalTok{,}
    \AttributeTok{y =} \StringTok{"Supporter signup rate"}\NormalTok{,}
    \AttributeTok{title =} \StringTok{"Issue{-}focused homepage increases signups by \textasciitilde{}7.7 percentage points"}\NormalTok{,}
    \AttributeTok{subtitle =} \StringTok{"Bars show mean signup rates with 95\% confidence intervals (n = 1,500 visitors)"}
\NormalTok{  ) }\SpecialCharTok{+}
  \FunctionTok{theme\_minimal}\NormalTok{(}\AttributeTok{base\_size =} \DecValTok{13}\NormalTok{) }\SpecialCharTok{+}
  \FunctionTok{theme}\NormalTok{(}
    \AttributeTok{legend.position =} \StringTok{"none"}\NormalTok{,}
    \AttributeTok{plot.title =} \FunctionTok{element\_text}\NormalTok{(}\AttributeTok{face =} \StringTok{"bold"}\NormalTok{, }\AttributeTok{size =} \DecValTok{14}\NormalTok{),}
    \AttributeTok{plot.subtitle =} \FunctionTok{element\_text}\NormalTok{(}\AttributeTok{size =} \DecValTok{11}\NormalTok{, }\AttributeTok{color =} \StringTok{"gray30"}\NormalTok{)}
\NormalTok{  )}
\end{Highlighting}
\end{Shaded}

\includegraphics{gv300-showcase_files/figure-latex/visualisation-1.pdf}

\begin{center}\rule{0.5\linewidth}{0.5pt}\end{center}

\subsection{What This Figure Shows}\label{what-this-figure-shows}

The visualisation above compares supporter signup rates between visitors
shown the control homepage (standard party messaging) and those shown
the issue-focused homepage.

\textbf{Key findings:}

\begin{enumerate}
\def\labelenumi{\arabic{enumi}.}
\tightlist
\item
  \textbf{Control group}: 9.6\% of visitors signed up as supporters
  (95\% CI: 7.8\% to 11.4\%)
\item
  \textbf{Treatment group}: 17.3\% of visitors signed up (95\% CI:
  15.1\% to 19.5\%)
\item
  \textbf{Average Treatment Effect (ATE)}: +7.7 percentage points
\end{enumerate}

The error bars (95\% confidence intervals) show that this difference is
statistically significant---they do not overlap, indicating the
treatment effect is unlikely to be due to random chance alone. In fact,
the confidence intervals are well-separated, with the entire control
interval below the entire treatment interval.

From a practical perspective, this 7.7 percentage point increase
represents an \textbf{80\% relative improvement} over the baseline
signup rate (9.6\% → 17.3\%). For a political party seeking to expand
its supporter base, this magnitude of effect could translate into
thousands of additional signups over time---a substantively meaningful
impact for campaign strategy.

\begin{center}\rule{0.5\linewidth}{0.5pt}\end{center}

\subsection{Model Comparison}\label{model-comparison}

To ensure robustness, I estimated the treatment effect using three
different approaches:

\begin{Shaded}
\begin{Highlighting}[]
\CommentTok{\# 1. Difference{-}in{-}means (ATE)}
\NormalTok{mean\_control }\OtherTok{\textless{}{-}} \FunctionTok{mean}\NormalTok{(party\_ab}\SpecialCharTok{$}\NormalTok{support\_signup[party\_ab}\SpecialCharTok{$}\NormalTok{treatment }\SpecialCharTok{==} \DecValTok{0}\NormalTok{])}
\NormalTok{mean\_treated }\OtherTok{\textless{}{-}} \FunctionTok{mean}\NormalTok{(party\_ab}\SpecialCharTok{$}\NormalTok{support\_signup[party\_ab}\SpecialCharTok{$}\NormalTok{treatment }\SpecialCharTok{==} \DecValTok{1}\NormalTok{])}
\NormalTok{ate\_diff }\OtherTok{\textless{}{-}}\NormalTok{ mean\_treated }\SpecialCharTok{{-}}\NormalTok{ mean\_control}

\CommentTok{\# 2. Linear Probability Model}
\NormalTok{lm\_model }\OtherTok{\textless{}{-}} \FunctionTok{lm}\NormalTok{(support\_signup }\SpecialCharTok{\textasciitilde{}}\NormalTok{ treatment, }\AttributeTok{data =}\NormalTok{ party\_ab)}

\CommentTok{\# 3. Logistic Regression}
\NormalTok{logit\_model }\OtherTok{\textless{}{-}} \FunctionTok{glm}\NormalTok{(support\_signup }\SpecialCharTok{\textasciitilde{}}\NormalTok{ treatment, }
                   \AttributeTok{data =}\NormalTok{ party\_ab, }
                   \AttributeTok{family =} \FunctionTok{binomial}\NormalTok{(}\AttributeTok{link =} \StringTok{"logit"}\NormalTok{))}

\CommentTok{\# Predicted probabilities from logit}
\NormalTok{p\_control\_logit }\OtherTok{\textless{}{-}} \FunctionTok{predict}\NormalTok{(logit\_model, }
                           \AttributeTok{newdata =} \FunctionTok{data.frame}\NormalTok{(}\AttributeTok{treatment =} \DecValTok{0}\NormalTok{), }
                           \AttributeTok{type =} \StringTok{"response"}\NormalTok{)}
\NormalTok{p\_treated\_logit }\OtherTok{\textless{}{-}} \FunctionTok{predict}\NormalTok{(logit\_model, }
                           \AttributeTok{newdata =} \FunctionTok{data.frame}\NormalTok{(}\AttributeTok{treatment =} \DecValTok{1}\NormalTok{), }
                           \AttributeTok{type =} \StringTok{"response"}\NormalTok{)}
\NormalTok{ate\_logit }\OtherTok{\textless{}{-}}\NormalTok{ p\_treated\_logit }\SpecialCharTok{{-}}\NormalTok{ p\_control\_logit}

\CommentTok{\# Display results}
\NormalTok{results }\OtherTok{\textless{}{-}} \FunctionTok{data.frame}\NormalTok{(}
  \AttributeTok{Method =} \FunctionTok{c}\NormalTok{(}\StringTok{"Difference{-}in{-}means"}\NormalTok{, }\StringTok{"Linear Probability Model"}\NormalTok{, }\StringTok{"Logistic Regression"}\NormalTok{),}
  \AttributeTok{ATE =} \FunctionTok{c}\NormalTok{(ate\_diff, }\FunctionTok{coef}\NormalTok{(lm\_model)[}\StringTok{"treatment"}\NormalTok{], ate\_logit)}
\NormalTok{)}
\FunctionTok{print}\NormalTok{(results)}
\end{Highlighting}
\end{Shaded}

\begin{verbatim}
##                             Method       ATE
##                Difference-in-means 0.0769346
## treatment Linear Probability Model 0.0769346
## 1              Logistic Regression 0.0769346
\end{verbatim}

All three methods produce nearly identical estimates
(\textbf{\textasciitilde0.077, or 7.7 percentage points}), which is
expected in a well-randomized experiment with a binary treatment and an
outcome probability far from 0 or 1. This convergence gives confidence
in the finding. The linear probability model coefficient tells us that
the issue-focused homepage increases signup probability by about 7.7
percentage points compared to the control.

\begin{center}\rule{0.5\linewidth}{0.5pt}\end{center}

\subsection{Linear Probability Model
Interpretation}\label{linear-probability-model-interpretation}

\begin{Shaded}
\begin{Highlighting}[]
\FunctionTok{summary}\NormalTok{(lm\_model)}
\end{Highlighting}
\end{Shaded}

\begin{verbatim}
## 
## Call:
## lm(formula = support_signup ~ treatment, data = party_ab)
## 
## Residuals:
##      Min       1Q   Median       3Q      Max 
## -0.17298 -0.17298 -0.09605 -0.09605  0.90395 
## 
## Coefficients:
##             Estimate Std. Error t value Pr(>|t|)    
## (Intercept)  0.09605    0.01284   7.482 1.24e-13 ***
## treatment    0.07693    0.01767   4.355 1.42e-05 ***
## ---
## Signif. codes:  0 '***' 0.001 '**' 0.01 '*' 0.05 '.' 0.1 ' ' 1
## 
## Residual standard error: 0.3416 on 1498 degrees of freedom
## Multiple R-squared:  0.0125, Adjusted R-squared:  0.01184 
## F-statistic: 18.97 on 1 and 1498 DF,  p-value: 1.422e-05
\end{verbatim}

The intercept (0.096) estimates the baseline signup probability for
visitors in the control group---about \textbf{9.6\%}. The treatment
coefficient (0.077) represents the Average Treatment Effect: being shown
the issue-focused homepage increases signup probability by \textbf{7.7
percentage points}. This is identical to the difference-in-means
calculation (\texttt{ate\_diff}), demonstrating that in a randomized
experiment with a binary treatment, OLS regression simply reproduces the
group mean difference.

\begin{center}\rule{0.5\linewidth}{0.5pt}\end{center}

\subsection{Heterogeneous Effects by Interest
Level}\label{heterogeneous-effects-by-interest-level}

The assignment required exploring whether treatment effects vary across
different types of visitors. I added a simulated covariate for political
interest:

\begin{Shaded}
\begin{Highlighting}[]
\CommentTok{\# Add simulated covariate}
\NormalTok{party\_ab}\SpecialCharTok{$}\NormalTok{high\_interest }\OtherTok{\textless{}{-}} \FunctionTok{rbinom}\NormalTok{(}\FunctionTok{nrow}\NormalTok{(party\_ab), }\AttributeTok{size =} \DecValTok{1}\NormalTok{, }\AttributeTok{prob =} \FloatTok{0.4}\NormalTok{)}

\CommentTok{\# Interaction model}
\NormalTok{lm\_het\_interest }\OtherTok{\textless{}{-}} \FunctionTok{lm}\NormalTok{(support\_signup }\SpecialCharTok{\textasciitilde{}}\NormalTok{ treatment }\SpecialCharTok{*}\NormalTok{ high\_interest, }\AttributeTok{data =}\NormalTok{ party\_ab)}
\FunctionTok{summary}\NormalTok{(lm\_het\_interest)}
\end{Highlighting}
\end{Shaded}

\begin{verbatim}
## 
## Call:
## lm(formula = support_signup ~ treatment * high_interest, data = party_ab)
## 
## Residuals:
##      Min       1Q   Median       3Q      Max 
## -0.18008 -0.18008 -0.10462 -0.08418  0.91582 
## 
## Coefficients:
##                          Estimate Std. Error t value Pr(>|t|)    
## (Intercept)              0.104623   0.016853   6.208 6.94e-10 ***
## treatment                0.075462   0.023051   3.274  0.00109 ** 
## high_interest           -0.020448   0.026021  -0.786  0.43210    
## treatment:high_interest  0.002863   0.035906   0.080  0.93646    
## ---
## Signif. codes:  0 '***' 0.001 '**' 0.01 '*' 0.05 '.' 0.1 ' ' 1
## 
## Residual standard error: 0.3417 on 1496 degrees of freedom
## Multiple R-squared:  0.01324,    Adjusted R-squared:  0.01126 
## F-statistic: 6.692 on 3 and 1496 DF,  p-value: 0.0001739
\end{verbatim}

\begin{Shaded}
\begin{Highlighting}[]
\CommentTok{\# Calculate treatment effects by subgroup}
\CommentTok{\# Low interest (high\_interest = 0)}
\NormalTok{te\_low }\OtherTok{\textless{}{-}} \FunctionTok{coef}\NormalTok{(lm\_het\_interest)[}\StringTok{"treatment"}\NormalTok{]}

\CommentTok{\# High interest (high\_interest = 1)  }
\NormalTok{te\_high }\OtherTok{\textless{}{-}} \FunctionTok{coef}\NormalTok{(lm\_het\_interest)[}\StringTok{"treatment"}\NormalTok{] }\SpecialCharTok{+} \FunctionTok{coef}\NormalTok{(lm\_het\_interest)[}\StringTok{"treatment:high\_interest"}\NormalTok{]}

\FunctionTok{data.frame}\NormalTok{(}
  \AttributeTok{Subgroup =} \FunctionTok{c}\NormalTok{(}\StringTok{"Low Interest"}\NormalTok{, }\StringTok{"High Interest"}\NormalTok{),}
  \AttributeTok{Treatment\_Effect =} \FunctionTok{c}\NormalTok{(te\_low, te\_high)}
\NormalTok{)}
\end{Highlighting}
\end{Shaded}

\begin{verbatim}
##        Subgroup Treatment_Effect
## 1  Low Interest       0.07546187
## 2 High Interest       0.07832492
\end{verbatim}

The interaction term (\texttt{treatment:high\_interest}) tells us
whether the effect of the issue-focused homepage differs between low-
and high-interest visitors. A positive coefficient would mean the
treatment works better for highly interested voters; a negative
coefficient would mean it works better for the less engaged. This kind
of heterogeneity analysis helps campaigns understand whether messaging
effects are universal or targeted.

\begin{center}\rule{0.5\linewidth}{0.5pt}\end{center}

\subsection{Why This Work Matters}\label{why-this-work-matters}

This analysis demonstrates several important concepts in quantitative
political science:

\begin{enumerate}
\def\labelenumi{\arabic{enumi}.}
\item
  \textbf{Causal inference with randomized experiments}: Random
  assignment allows us to estimate treatment effects without confounding
  bias, providing credible evidence about what works in political
  communication.
\item
  \textbf{The potential outcomes framework}: Understanding that each
  unit has two potential outcomes, but we only observe one, is
  fundamental to thinking causally about any intervention.
\item
  \textbf{Multiple estimation approaches}: Seeing that
  difference-in-means, linear regression, and logistic regression all
  point to the same conclusion builds confidence in results and shows
  methodological flexibility.
\item
  \textbf{Uncertainty communication}: Including confidence intervals in
  visualisations helps readers understand the precision of estimates,
  not just point predictions---crucial for responsible data
  communication.
\item
  \textbf{Practical significance}: Translating statistical results into
  substantive interpretations (e.g., \textbf{``80\% relative
  improvement''}) makes findings accessible to non-technical audiences
  like campaign strategists.
\item
  \textbf{Ethical awareness}: The assignment prompted reflection on
  whether optimizing political messages through experimentation serves
  democratic values or merely manipulates voters---a tension between
  technical efficiency and ethical responsibility.
\end{enumerate}

\begin{center}\rule{0.5\linewidth}{0.5pt}\end{center}

\subsection{Ethical Reflection}\label{ethical-reflection}

While this analysis focused on technical estimation, the assignment also
prompted reflection on the ethics of political experimentation. Key
considerations include:

\begin{itemize}
\tightlist
\item
  \textbf{Informed consent}: Website visitors typically don't know
  they're in an experiment, yet messaging could influence political
  preferences or engagement.
\item
  \textbf{Manipulation risks}: Issue-focused messaging may deliberately
  amplify fear, anger, or prejudice (e.g., around immigration), risking
  harm to vulnerable groups and polarisation.
\item
  \textbf{Fairness and representation}: If some groups are
  systematically more exposed to persuasive messages, experiments could
  widen inequalities in political voice.
\item
  \textbf{Data privacy}: Tracking signup behaviour may involve sensitive
  political data; parties must follow data protection rules and minimise
  data collection.
\item
  \textbf{Democratic implications}: Political communication experiments
  can be used to optimise persuasion rather than promote informed
  choice, potentially undermining deliberation.
\end{itemize}

These ethical dimensions remind us that statistical tools, however
powerful, must be deployed thoughtfully in political contexts.

\begin{center}\rule{0.5\linewidth}{0.5pt}\end{center}

\subsection{Source Code}\label{source-code}

The complete R Markdown source for this analysis is available in my
\href{https://github.com/2007263/gv300-website}{GV300 assignments
repository}.

\end{document}
